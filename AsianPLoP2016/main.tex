% This is "sig-alternate.tex" V2.1 April 2013
% This file should be compiled with V2.5 of "sig-alternate.cls" May 2012
%
% This example file demonstrates the use of the 'sig-alternate.cls'
% V2.5 LaTeX2e document class file. It is for those submitting
% articles to ACM Conference Proceedings WHO DO NOT WISH TO
% STRICTLY ADHERE TO THE SIGS (PUBS-BOARD-ENDORSED) STYLE.
% The 'sig-alternate.cls' file will produce a similar-looking,
% albeit, 'tighter' paper resulting in, invariably, fewer pages.
%
% ----------------------------------------------------------------------------------------------------------------
% This .tex file (and associated .cls V2.5) produces:
%       1) The Permission Statement
%       2) The Conference (location) Info information
%       3) The Copyright Line with ACM data
%       4) NO page numbers
%
% as against the acm_proc_article-sp.cls file which
% DOES NOT produce 1) thru' 3) above.
%
% Using 'sig-alternate.cls' you have control, however, from within
% the source .tex file, over both the CopyrightYear
% (defaulted to 200X) and the ACM Copyright Data
% (defaulted to X-XXXXX-XX-X/XX/XX).
% e.g.
% \CopyrightYear{2007} will cause 2007 to appear in the copyright line.
% \crdata{0-12345-67-8/90/12} will cause 0-12345-67-8/90/12 to appear in the copyright line.
%
% ---------------------------------------------------------------------------------------------------------------
% This .tex source is an example which *does* use
% the .bib file (from which the .bbl file % is produced).
% REMEMBER HOWEVER: After having produced the .bbl file,
% and prior to final submission, you *NEED* to 'insert'
% your .bbl file into your source .tex file so as to provide
% ONE 'self-contained' source file.
%
% ================= IF YOU HAVE QUESTIONS =======================
% Questions regarding the SIGS styles, SIGS policies and
% procedures, Conferences etc. should be sent to
% Adrienne Griscti (griscti@acm.org)
%
% Technical questions _only_ to
% Gerald Murray (murray@hq.acm.org)
% ===============================================================
%
% For tracking purposes - this is V2.0 - May 2012

\documentclass{sig-alternate-05-2015}
\usepackage[utf8]{inputenc}

\begin{document}

% Copyright
\setcopyright{acmcopyright}
%\setcopyright{acmlicensed}
%\setcopyright{rightsretained}
%\setcopyright{usgov}
%\setcopyright{usgovmixed}
%\setcopyright{cagov}
%\setcopyright{cagovmixed}


% DOI
\doi{10.475/123_4}

% ISBN
\isbn{123-4567-24-567/08/06}

%Conference
\conferenceinfo{AsianPLoP 2016}{February 24-26, Taiwan.}

%\acmPrice{\$15.00}

%
% --- Author Metadata here ---
\conferenceinfo{WOODSTOCK}{'97 El Paso, Texas USA}
%\CopyrightYear{2007} % Allows default copyright year (20XX) to be over-ridden - IF NEED BE.
%\crdata{0-12345-67-8/90/01}  % Allows default copyright data (0-89791-88-6/97/05) to be over-ridden - IF NEED BE.
% --- End of Author Metadata ---

\title{Browser Infrastrcture Pattern}
%\subtitle{[Extended Abstract]
%\titlenote{A full version of this paper is available as
%\textit{Author's Guide to Preparing ACM SIG Proceedings Using
%\LaTeX$2_\epsilon$\ and BibTeX} at


\numberofauthors{3} %  in this sample file, there are a *total*
% of EIGHT authors. SIX appear on the 'first-page' (for formatting
% reasons) and the remaining two appear in the \additionalauthors section.
%

\author{
\alignauthor
Paulina Silva\\
  \affaddr{Departamento de Informática}\\
  \affaddr{Universidad Técnica Federico Santa María}\\
  \affaddr{Valparaíso, Chile}\\
  \email{pasilva@alumnos.inf.utfsm.cl}
% 2nd. author
\alignauthor
Raúl Monge\\
  \affaddr{Departamento de Informática}\\
  \affaddr{Universidad Técnica Federico Santa María}\\
  \affaddr{Valparaíso, Chile}\\
  \email{rmonge@inf.utfsm.cl}
% 3rd. author
\alignauthor 
Eduardo Fernandez\\
  \affaddr{Department of Computer \(\&\) Electrical Engineering and Computer Science}\\
  \affaddr{Florida Atlantic University}\\
  \affaddr{Florida, USA}\\
  \email{ed@cse.fau.edu}
}

\maketitle
\begin{abstract}
Currently a lot of software developments create systems that are connected to the Internet, which allows to add functionality within a system and facilities to their \textit{Stakeholders}. This leads to depend in a \textit{web client}, as the \textit{Web Browser}, which allows access to services, data or operations that the system delivers. Nevertheless, the Internet influences the attack surface of the new system, and unfortunately many stakeholders and developers are not aware of the risks they are exposed. The lack of Security Education in Software developers of a project, the low and scattered documentation of each browser (and standardization), could become a great flaw in big architectural developments which depends on the browser to do their services. A Reference Architecture of the \textit{Web Browser}, using Architectural Patterns, could be a base for understanding the security mechanisms and its architecture, which interacts with a bigger web system. This would give an unification of ideas and terminology, giving a holistic view regardless the implementation details for both the browser and the system it communicates to. We developed a Browser Infrastrcture Pattern which describes the infrastructure to allow the communication between a Web Client and Server in the Internet. With this work we propose an Architectural Pattern as the first piece of our Reference Architecture for the Web Browser.
\end{abstract}


%
% The code below should be generated by the tool at
% http://dl.acm.org/ccs.cfm
% Please copy and paste the code instead of the example below. 
%
\begin{CCSXML}
<ccs2012>
 <concept>
  <concept_id>10010520.10010553.10010562</concept_id>
  <concept_desc>Computer systems organization~Embedded systems</concept_desc>
  <concept_significance>500</concept_significance>
 </concept>
 <concept>
  <concept_id>10010520.10010575.10010755</concept_id>
  <concept_desc>Computer systems organization~Redundancy</concept_desc>
  <concept_significance>300</concept_significance>
 </concept>
 <concept>
  <concept_id>10010520.10010553.10010554</concept_id>
  <concept_desc>Computer systems organization~Robotics</concept_desc>
  <concept_significance>100</concept_significance>
 </concept>
 <concept>
  <concept_id>10003033.10003083.10003095</concept_id>
  <concept_desc>Networks~Network reliability</concept_desc>
  <concept_significance>100</concept_significance>
 </concept>
</ccs2012>  
\end{CCSXML}

\ccsdesc[500]{Computer systems organization~Embedded systems}
\ccsdesc[300]{Computer systems organization~Redundancy}
\ccsdesc{Computer systems organization~Robotics}
\ccsdesc[100]{Networks~Network reliability}




\keywords{Web Browser, Web Client, Modular Architecture, Browser Architecture, Reference Architecture, Browser Infrastructure pattern}

\section*{Introduction}

\section*{Background}


\section*{Related Work}

\section*{Browser Infrastructure Pattern}

  \subsection*{Intent}
  The Browser Infrastructure Pattern allows the request of a web resource in the Internet to a \textbf{Browser User}, which is a user who uses a Browser within a Host. The Pattern lets visualize the communication between the components that make the Web Browser and the Provider (i.e, a Server), to whom the request is made.

  \subsection*{Example}
  Within the Host it is possible a lack of resources that a Host user may need. The request of external services or resources is the main reason of the Internet existence. This kind of task it is possible to do in a lot of ways, it all depends on what the Provider wish to deliver to others.
  
  \subsection*{Context}

  \subsection*{Problem}

  \subsection*{Solution}

    \subsubsection*{Structure}

    \subsubsection*{Dynamics}

  \subsection*{Implementation}

  \subsection*{Consequences}

  \subsection*{Example Resolved}

  \subsection*{Known Uses}

  \subsection*{Related Patterns}

\section*{Conclusions}

%ACKNOWLEDGMENTS are optional
%\section{Acknowledgments}


\bibliographystyle{abbrv}
\bibliography{refTodas}  

\subsection*{References}

\end{document}
